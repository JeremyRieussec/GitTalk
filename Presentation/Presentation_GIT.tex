\documentclass{beamer}

\usepackage[utf8]{inputenc}
\usepackage[english]{babel}

% -------- Mise en page----------
\usepackage{color}
\usepackage{multicol}
\usepackage{csquotes}
%\setlength{\columnseprule}{0.4pt}
\setlength{\columnsep}{3em}
\usepackage{setspace}

\usepackage{microtype}
\DisableLigatures{}

\usepackage{bbding}

\usepackage{lscape}

% Asymptote
\usepackage[inline]{asymptote}
\usepackage{icomma}

\usepackage{xcolor}
\usepackage{caption}

\usepackage{array,multirow,makecell}
\newcolumntype{R}[1]{>{\raggedleft\arraybackslash}b{#1}}
\newcolumntype{L}[1]{>{\raggedright\arraybackslash}b{#1}}
\newcolumntype{C}[1]{>{\centering\arraybackslash}b{#1}}


\usepackage{enumerate}

%\usepackage{framed}
%\usepackage[framed]{ntheorem}

% ---------  Packages de maths------------
\usepackage{amsmath,amssymb}
%\usepackage{amsthm}
\usepackage{amsfonts}
\usepackage{mathrsfs}
\usepackage{bbm}

\usepackage{animate}
\usepackage{hyperref}
\usepackage{color}
\usepackage{ifpdf}

%------------- Packages pour les graphiques et dessins ----------------
\usepackage{graphicx}

\usepackage{numprint}

\usepackage{xcolor}
\usepackage{caption}

% ----------- Macros Maths ----------------------

\newcommand*\N{\mathbb{N}}
\newcommand*\Z{\mathbb{Z}}
\newcommand*\Q{\mathbb{Q}}
\newcommand*\R{\mathbb{R}}
\newcommand*\C{\mathbb{C}}

\newcommand*\Rd{\mathbb{R}^d}
\newcommand*\Rm{\mathbb{R}^m}
\newcommand*\Rp{\mathbb{R}^p}


\newcommand*\Mat[2]{\mathbb{R}^{#1 \times{} #2}}

\newcommand*\suite[3]{ \{ #1_{#2} \}_{#2 \in{} #3} }

\newcommand*\expo[1]{ \mathrm{e}^{#1} }
\newcommand*\e{ \mathrm{e} }

\usepackage{epic,bez123}
\usepackage{floatflt}% package for floatingfigure environment
\usepackage{wrapfig}% package for wrapfigure environment

\usepackage{pstricks-add}

\usepackage{listings}

\usetheme{Warsaw}

\title{About Git and GitHub}
\subtitle{}
\author{Jérémy Rieussec}
\institute{University of Montreal}
\date{\today}

\begin{document}
	
	\setbeamercovered{transparent}	
	
	% Parametres pour la gestion des captions de figure
	%\setbeamertemplate{caption}[default]
	\setbeamertemplate{caption}[numbered]
	
	\begin{frame}[plain]
		\titlepage{}
	\end{frame}		

	\begin{frame}[plain]
		\frametitle{Ressources}
		\small		
		\url{https://www.coursera.org/learn/version-control-with-git}
	\end{frame}



    \begin{frame}[plain]
		\tableofcontents
	\end{frame}	

	
	\section{What is Git?}
    \begin{frame}[plain]
		\frametitle{Git is a \textbf{version control software} }
		\begin{enumerate}
			\item content
			\begin{itemize}
				\item track history of project
			\end{itemize}
			\item teams
			\begin{itemize}
				\item collaboration, 
				\item workflows,  $\dots$
			\end{itemize}
			\item agility
			\begin{itemize}
				\item test
				\item fix or undo changes
			\end{itemize}
		\end{enumerate}
	\end{frame}	

	\begin{frame}[plain]
		\frametitle{Distributed version control system}
		
		\includegraphics[scale=0.48]{Images/Distributed_version_control.png}
	\end{frame}

	\begin{frame}[plain]
		\frametitle{What is a Git repository?}
		A series of \textit{snapshots}, also called \textbf{COMMITS}.
		\begin{center}
			\includegraphics[scale=0.7]{Images/commits.png}

		\end{center}
	\end{frame}

	\begin{frame}[plain]
		\frametitle{How to use Git?}
		\begin{enumerate}
			\item Command line: Git Bash \\
			\item Git UI \\
		\end{enumerate}		

		~\linebreak \\

		\textbf{Advise:} master command line 
		\begin{itemize}
			\item good for deep understanding
			\item broadly used in industry
			\item automatable
		\end{itemize}
	\end{frame}

	\section{Setup}
    \begin{frame}[plain]
		\frametitle{Work configuration I use and find quite good}
		\begin{itemize}
			\item GIT 
			\item Pycharm: for quick editions, solving conflicts, $\dots$
			\item Visual code: for project overview
		\end{itemize}
		
	\end{frame}	

	\begin{frame}[plain]
		\frametitle{Setting GIT}

		\begin{itemize}
			\item see configuration: \\
			\begin{center}
				git config --list
			\end{center}
			\item user name: \\
			\begin{center}
				git config --global user.name ``Hello''
			\end{center}
			\item email: \\
			\begin{center}
				git config --global user.email ``Hello.world@gmail.com''
			\end{center}
			\item editor: \\
			\begin{center}
				git config --global user.editor ``pycharm64.exe''
			\end{center}
		\end{itemize}		
	\end{frame}

	\begin{frame}[plain]
		\frametitle{Getting help}
		
		\begin{center}
			git --help  \\
			git -h
		\end{center}
		\begin{center}
			\includegraphics[scale=0.45]{Images/help_general.png}
		\end{center}
	
	\end{frame}

	\begin{frame}[plain]
		\frametitle{Getting help on a specific command}
		
		\begin{center}
			git --help $<\text{command\_name}>$ \\
			git -h $<\text{command\_name}>$
		\end{center}
		
		\textit{Example:} git --help config
	
	\end{frame}

	\section{Working directory}
    \begin{frame}[plain]
		\frametitle{Git repository}

		\begin{center}
			\includegraphics[scale=0.5]{Images/working_tree.png}
		\end{center}
	\end{frame}

	\begin{frame}[plain]
		\frametitle{Git repository}

		\begin{center}
			\includegraphics[scale=0.5]{Images/working_tree_1.png}
		\end{center}
	\end{frame}	

	\begin{frame}[plain]
		\frametitle{Git repository}

		\begin{center}
			\includegraphics[scale=0.5]{Images/working_tree_2.png}
		\end{center}
	\end{frame}	
	
	\begin{frame}[plain]
		\frametitle{Git repository}

		\begin{center}
			\includegraphics[scale=0.6]{Images/working_tree_3.png}
		\end{center}
	\end{frame}	

	\begin{frame}[plain]
		\frametitle{Git repository}

		\begin{center}
			\includegraphics[scale=0.5]{Images/working_tree_4.png}
		\end{center}
	\end{frame}	

	\section{Adding, commits and references}

	\begin{frame}[plain]
		\frametitle{Creating and first commits git repository}

		\begin{enumerate}
			\item git init $\rightarrow$ creation ./git (go check)
			\begin{center}
				\includegraphics[scale=0.6]{Images/init.png}
			\end{center}
			\item git status
			\begin{center}
				\includegraphics[scale=0.5]{Images/status.png}
			\end{center}
			\item New modifications are \textcolor{red}{untracked} $\rightarrow$ changes must be \textcolor{green}{staged} to be commited.
			\begin{itemize}
				\item git add \textit{filename} 
				\item git add . (to add all files)
			\end{itemize}
			\item git diff
			
		\end{enumerate}	
	\end{frame}

	\begin{frame}[plain]
		\frametitle{Committing changes}

		\begin{center}
			git commit -m ``\textit{commit message}''
		\end{center}

		\begin{center}
			
		\end{center}
		
	
	\end{frame}

	\begin{frame}[plain]
		\frametitle{.gitignore}
		
		\begin{center}
			\includegraphics[scale=0.5]{Images/gitignore.png}
		\end{center}
		
	
	\end{frame}

	\begin{frame}[plain]
		\frametitle{History of commits}
			 \begin{center}
				git log 
				\begin{itemize}
					\item git log --oneline
					\item git log --oneline --graph
					\item git log --oneline --graph --all
				\end{itemize}

			 \end{center}

			 Every commit has an ID 
			 \begin{itemize}
				 \item 40-character hexadecimal string
				 \item SHA-1 values: small changes in content lead to big differences in value
			 \end{itemize}

	\end{frame}

	\begin{frame}[plain]
		\frametitle{References}
		User-friendly name that points to 
		\begin{itemize}
			\item a commit SHA-1 hash
			\item another reference: known as \textit{symbolic reference}
		 \end{itemize}
		\begin{center}
			\includegraphics[scale=0.7]{Images/git_log.png}
		\end{center}
				 
	\end{frame}

	\begin{frame}[plain]
		\frametitle{Branch labels}
		\begin{center}
			\includegraphics[scale=0.7]{Images/master_ref.png}
		\end{center}
		\begin{itemize}
			\item Points to the most recent commit in the branch, \textit{tip of the branch}.
			\item implemented as a reference
		\end{itemize}
	\end{frame}

	\begin{frame}[plain]
		\frametitle{HEAD}
		\begin{center}
			\includegraphics[scale=0.5]{Images/HEAD.png}
		\end{center}

		\begin{itemize}
			\item A reference to the current commit 
			\item Usually points to the label of the current branch
			\item one HEAD per repository
		\end{itemize}
		
		\begin{center}
			\includegraphics[scale=0.5]{Images/HEAD_master.png}
		\end{center}
	\end{frame}

	\begin{frame}[plain]
		\frametitle{Detached HEAD}
		HEAD points to the current commit, but we naviguate in commits with:
		\begin{center}
			git checkout \textit{commitID}
		\end{center}
		\begin{center}
			\includegraphics[scale=0.6]{Images/detached_HEAD.png}
		\end{center}
	
	\end{frame}

	\begin{frame}[plain]
		\frametitle{Tags}
		Reference/label attached to a specific commit.
		\begin{center}
			\includegraphics[scale=0.5]{Images/tag.png}
		\end{center}
		
	\end{frame}

	\begin{frame}[plain]
		\frametitle{Tags (continued)}
		
		To tag a commit:
		\begin{itemize}
			\item git tag $<$tagname$>$ $[<$commit$>]$
			\item $<$commit$>$ defaults to HEAD
			\item git tag $-$a $-$m ``includes feature 2'' v2.0 $\rightarrow$ annotate a tag with message
		\end{itemize}
		To use tags:
		\begin{itemize}
			\item git tag $\rightarrow$ show all tags
			\item git show \textit{tag} $\rightarrow$ tags can be used in git caommands
		\end{itemize}
	\end{frame}

	\begin{frame}[plain]
		\frametitle{Exercice}
		\begin{itemize}
			\item git log / git show HEAD / git show master
			\item check .git before checkout
			\item check .git after checkout
		\end{itemize}
	\end{frame}

	\section{Branching}
    \begin{frame}[plain]
		\frametitle{What is a branch?}
		A \textit{branch} is the set of commits that trace back to the project's first commit
		\begin{center}
			\includegraphics[scale=0.6]{Images/What_is_branch.png}
		\end{center}
		\textit{Example:}
		\begin{itemize}
			\item master: A and B
			\item featureX: A, B and C
		\end{itemize}
	\end{frame}	

	\begin{frame}[plain]
		\frametitle{Advantages branches}
		
		\begin{itemize}
			\item Fast and easy to create 
			\item Enable experimentation
					\begin{itemize}
						\item delete or, 
						\item merge
					\end{itemize}
			\item team development
		\end{itemize}
	
	\end{frame}

	\begin{frame}[plain]
		\frametitle{Topic and long-lived branches}
		\begin{itemize}
			\item Topic: for fixing bugs, adding features, $\dots$ \\
					$\rightarrow$ very short lived 
			\item Long-lived \\
				master, develop, release, $\dots$
		\end{itemize}
		\begin{center}
			\includegraphics[scale=0.6]{Images/Topic_Branch.png}
		\end{center}

		To view list of banches:
			\begin{center}
				git branch
			\end{center}
		
	
	\end{frame}

	\begin{frame}[plain]
		\frametitle{Creating branch}
		
		\begin{center}
			git branch $<\text{name}>$
		\end{center}
		
		\begin{center}
			\includegraphics[scale=0.7]{Images/create_branch.png}
		\end{center}
	
	\end{frame}

	\begin{frame}[plain]
		\frametitle{Checkout branch}
		
		\begin{center}
			git checkout $<\text{branch-name}>$
		\end{center}

		\begin{enumerate}
			\item updates the HEAD reference
			\item updates the working treewith the commit's files
		\end{enumerate}

		\begin{center}
			\includegraphics[scale=0.5]{Images/checkout.png}
		\end{center}
	\end{frame}

	\begin{frame}[plain]
		\frametitle{Back to detached HEAD}
		\begin{enumerate}
			\item git checkout $<\text{commitID}>$
			\item git checkout -b $<\text{branch-name}>$ $<\text{commitID}>$
		\end{enumerate}
		\begin{center}
			\includegraphics[scale=0.65]{Images/detached_HEAD_branch.png}
		\end{center}
		
	\end{frame}

	\begin{frame}[plain]
		\frametitle{Delete branch, happy situation}
	
		\begin{center}
			git branch -d $<\text{branch-name}>$
		\end{center}
		
		\begin{center}
			\includegraphics[scale=0.6]{Images/delete_branch.png}
		\end{center}
	\end{frame}

	\begin{frame}[plain]
		\frametitle{Delete branch, dangling commits}
		

		\begin{center}
			git branch -D $<\text{branch-name}>$
		\end{center}
		
		\begin{center}
			\includegraphics[scale=0.6]{Images/delete_branch_D.png}
		\end{center}

		Consequence: dangling commit that will be garbage collected.
	\end{frame}

	\begin{frame}[plain]
		\frametitle{Creating branch, exercice}
		
		Two possilities:
		\begin{enumerate}
			\item two command-line
				\begin{center}
					git branch $<\text{branch-name}>$ \\
					git checkout $<\text{branch-name}>$
				\end{center}
			\item one command-line
				\begin{center}
					git checkout -b $<\text{branch-name}>$
				\end{center}
		\end{enumerate}
	
		\textit{Exercice:}
		\begin{enumerate}
			\item create new branch
			\item add / commit different files on each branch 
			\item observe changing working tree with checkout
		\end{enumerate}
	
	\end{frame}

	\section{Merging}

	\begin{frame}[plain]
		\frametitle{Merging}
		\begin{center}
			\includegraphics[scale=0.6]{Images/merge.png}
		\end{center}
		
		Main types of merge:
		\begin{enumerate}
			\item Fast-forward
			\item Merge commit
			\item (Squash merge*)
			\item (Rebase*)
		\end{enumerate}
	
	\end{frame}

	\begin{frame}[plain]
		\frametitle{Fast-forward}
		
		The master branch label is moved to commit C
		\begin{center}
			\includegraphics[scale=0.58]{Images/fast-forward.png}
		\end{center}
	\end{frame}

	\begin{frame}[plain]
		\frametitle{Fast-forward NOT possible}
		
		If label moved to commit C, then commit D is dangling and work is lost.
		\begin{center}
			\includegraphics[scale=0.5]{Images/fast-forward_not_possible.png}
		\end{center}
		
		\textit{Note:} in this situation, it will be a Merge commit
	\end{frame}

	\begin{frame}[plain]
		\frametitle{Fast-forward merging steps}
		
		\begin{enumerate}
			\item git checkout master
			\item git merge featureX
				\begin{center}
					$\rightarrow$ defaults to attempting a fast-forward merge
				\end{center}
			\item git branch -d featureX
		\end{enumerate}

		\begin{center}
			\includegraphics[scale=0.5]{Images/ff_merge_steps.png}
		\end{center}
	\end{frame}

	\begin{frame}[plain]
		\frametitle{Merge commit}
		
		\begin{itemize}
			\item Combines the commits at the tip of the branches
			\item Places the results in the merge commit
			\item M has two parents
		\end{itemize}
		
		\begin{center}
			\includegraphics[scale=0.6]{Images/merge_commit.png}
		\end{center}
	
	\end{frame}


	\begin{frame}[plain]
		\frametitle{Merge commit steps}
		\begin{enumerate}
			\item git checkout master
			\item git merge featureX
				\begin{itemize}
					\item accept or modify the message
					\item even if fast-forwardable, you can force merge commit \\
					\begin{center}
						git merge \textbf{--no-ff} featureX
					\end{center}
				\end{itemize}
			\item git branch -d featureX
		\end{enumerate}
		
		\begin{center}
			\includegraphics[scale=0.5]{Images/merge_commit_no_conflicts.png}
		\end{center}
	
	\end{frame}

	\begin{frame}[plain]
		\frametitle{Conflict!!!}
	
		\begin{center}
			\includegraphics[scale=0.45]{Images/merge_commit_conflict.png}
		\end{center}
	
	\end{frame}

	\begin{frame}[plain]
		\frametitle{Merge commit: Different hunks}
		
		\begin{center}
			\includegraphics[scale=0.3]{Images/merge_conflicts_diff_hunks.png}
		\end{center}
		
	
	\end{frame}

	\begin{frame}[plain]
		\frametitle{Small, frequent merges are the easiest}
		\begin{center}
			\includegraphics[scale=0.5]{Images/small_merges.png}
		\end{center}
	
	\end{frame}

	\begin{frame}[plain]
		\frametitle{Merge commit situation}
	
		\begin{center}
			\includegraphics[scale=0.6]{Images/merge_situation.png}
		\end{center}
	
	\end{frame}

	\begin{frame}[plain]
		\frametitle{Merge commit steps}

		\begin{enumerate}
			\item git checkout master 
			\item git merge featureX
			\item \begin{enumerate}
				\item CONFLICT $\rightarrow$ Both FileA.txt were Modified
				\item files with conflicts are \textbf{modified by Git} and placed in working tree
			\end{enumerate}
			\item fix FileA.txt
			\item git add FileA.txt
			\item git commit -m ``$\ldots$''
			\item git branch -d featureX
		\end{enumerate}
	\end{frame}

	\begin{frame}[plain]
		\frametitle{Display conflict}
		
		\begin{center}
			\includegraphics[scale=0.6]{Images/display_conflict.png}
		\end{center}
		
	\end{frame}

	\section{Remote Repositories}
	\begin{frame}[plain]
		Two famous Hosted options:
		\begin{center}
			\includegraphics[scale=0.2]{Images/GitHub-Logo.png}
		\end{center}
		and
		\begin{center}
			\includegraphics[scale=0.2]{Images/Bitbucket_Logo.png}
		\end{center}
		
	\end{frame}

	\begin{frame}[plain]
		\frametitle{Creating and cloning}
		
		\begin{enumerate}
			\item Create a repository in GitHub 
			\item Clone repository locally 
				\begin{itemize}
					\item git clone $<$url$>$
					\item git clone -b $<\text{branch-name}>$ $<$url$>$
				\end{itemize}
		\end{enumerate}
	\end{frame}

	\begin{frame}[plain]
		\frametitle{Information on remote repo}
		git remote -v 
		\begin{center}
			\includegraphics[scale=0.7]{Images/remote_verbose.png}
		\end{center}

		git remote show origin
		\begin{center}
			\includegraphics[scale=0.7]{Images/remote_show.png}
		\end{center}

		git remote add $<\text{name}>$ $<\text{url}>$
	\end{frame}

	\begin{frame}[plain]
		\frametitle{Tracking branch}
	
		\begin{center}
			\includegraphics[scale=0.58]{Images/tracking_branch.png}
		\end{center}
	\end{frame}

	\begin{frame}[plain]
		\frametitle{Tracking branches, related but decoupled}
		
		\begin{center}
			\includegraphics[scale=0.55]{Images/tracking_related.png}
		\end{center}
		
	\end{frame}

	\begin{frame}[plain]
		\frametitle{Push, pull and fetch}
	
		\begin{enumerate}
			\item git fetch $<\text{remote}>$ $<\text{branch-name}>$
			\item git push $<\text{remote}>$ $<\text{branch-name}>$
			\item git pull $<\text{remote}>$ $<\text{branch-name}>$ 
		\end{enumerate}
	
	\end{frame}

	\begin{frame}[plain]
		\frametitle{Fetch}
	
		\begin{center}
			\includegraphics[scale=0.6]{Images/fetch.png}
		\end{center}
	
	\end{frame}

	\begin{frame}[plain]
		\frametitle{Pull with fast-forward}
		
		\begin{center}
			\includegraphics[scale=0.6]{Images/pull_ff.png}
		\end{center}
		
	\end{frame}

	\begin{frame}[plain]
		\frametitle{Pull with merge commit}
	
		\begin{center}
			\includegraphics[scale=0.6]{Images/pull_merge.png}
		\end{center}	
	
	\end{frame}

	\section{Rebasing}
	\begin{frame}[plain]
		\frametitle{Rebasing}
	
		\begin{center}
			\includegraphics[scale=0.5]{Images/rebase.png}
		\end{center}
	
	\end{frame}

	\begin{frame}[plain]
		\frametitle{Rebase is a merge}
	
		\begin{center}
			\includegraphics[scale=0.55]{Images/rebase_merge_conflict.png}
		\end{center}
	
	\end{frame}

	\begin{frame}[plain]
		\frametitle{Steps}
		
		\begin{enumerate}
			\item git checkout featureX
			\item git rebase master
				\begin{enumerate}
					\item CONFLICT
				\end{enumerate}
			\item git status
				\begin{enumerate}
					\item Both modified fileA.txt
				\end{enumerate}
			\item Fix fileA.txt
			\item git add fileA.txt
			\item git rebase --continue
		\end{enumerate}
		
	
	\end{frame}

	\begin{frame}[plain]
		\frametitle{Merge vs Rebase}
	
		\begin{center}
			\includegraphics[scale=0.5]{Images/diff_merge_rebase.png}
		\end{center}
	
	\end{frame}


	\begin{frame}[plain]
		\frametitle{Rewritting history with Interactive rebase}
	
		\begin{center}
			\includegraphics[scale=0.6]{Images/rebase-int.png}
		\end{center}
	
	\end{frame}

	\begin{frame}[plain]
		\frametitle{Example}
	
		\begin{center}
			\includegraphics[scale=0.5]{Images/squash_diff.png}
		\end{center}
	
	\end{frame}


	\section{Pull requests}
	\begin{frame}[plain]
		\frametitle{When to create a pull request?}
		\begin{enumerate}
			\item When the branch is created
			\item When you want comments on the branch
			\item When the branch is ready for review/merging
		\end{enumerate}
	\end{frame}
\end{document}